\chapter{Testovanie}
Testovanie aplikácie je nevyhnutnou súčasťou pre potvrdenie, že implementovaná funkcionalita funguje podľa špecifikácie. Táto kapitola popisuje jednotkové a integračné testy, ktoré boli použité na otestovanie správnej funkcionality aplikácie. Následne sa kapitola venuje procesu kontroly a overenia správnosti importu existujúcich dát do novej databázy.

\section{Jednotkové testy}
Jednotkové testy sú testy, ktoré testujú jednotlivé komponenty softvéru. Účelom týchto testov je overiť, či každý komponent softvéru funguje tak, ako je navrhnutý. Jednotka je najmenšia testovateľná súčasť akéhokoľvek softvéru. Zvyčajne má jeden alebo niekoľko vstupov a zvyčajne jeden výstup \cite{co-je-unit-testing}.

Nakoľko biznis logika tejto aplikácie nie je rozsiahla, boli namiesto jednotkových testov vytvorené integračné testy.

\section{Integračné testy}
Integračné testovanie je úroveň testovania aplikácie, pri ktorej sa jednotlivé komponenty kombinujú a testujú ako skupina. Účelom tejto úrovne testovania je odhaliť poruchy v interakcii medzi integrovanými komponentami \cite{co-su-integracne-testy}.

Samotné integračné testy sa zameriavajú na testovanie komunikácie medzi klientom a serverom, pri ktorých sa overuje aktuálna a očakávaná odpoveď servera. Počas behu jednotlivých testov dochádza k overeniu autentifikácie používateľa, prístupu do databázy, respektíve overeniu práv používateľa k danej akcii.

\pagebreak

Príkazom \mintinline{php}{php artisan test} sa spustia všetky implementované testy. Tento príkaz je potrebné spustiť z hlavného adresára webovej aplikácie.

\begin{figure}[H]
\begin{minted}[xleftmargin=1.8em,linenos]{php}
public function testUserDelete() {
    // Create new user
    $user = factory(User::class)->create();

    // Deleting unauthenticated should result in 401
    $this->deleteJson(
        route('users.destroy',
        ['user' => $user->id])
    )->assertStatus(401);

    // Login as user
    $this->loginAsUser();

    // Deleting as user should result in 403
    $this->deleteJson(
        route('users.destroy',
        ['user' => $user->id])
    )->assertStatus(403);

    // Login as admin
    $this->loginAsAdmin();

    // Now the deletion should be alright
    $this->deleteJson(
        route('users.destroy',
        ['user' => $user->id])
    )->assertStatus(204);
}
\end{minted}
\caption[Ukážka integračného testu zmazania používateľa]
{Ukážka integračného testu zmazania používateľa}
\label{user-delete-test-code}
\end{figure}

Na ukážke testu \ref{user-delete-test-code} je zobrazený priebeh testu zmazania používateľa. V tomto teste je využitá kontrola prihlásenia a následná kontrola práv. Ak je používateľ odhlásený, respektíve nemá dostatočné práva pre zmazanie používateľa, server by mal vrátiť odpoveď s HTTP status kódom 401, resp. 403. Po prihlásení sa ako administrátor by malo byť možné zmazať vytvoreného používateľa poslaním DELETE požiadavky na server.

\pagebreak 

Na podobnom princípe sú postavené ostatné integračné testy. Integračné testy boli vytvorené pre všetky entity, ktoré sú dostupné cez API servera. Tieto testy sa nachádzajú v adresári \mintinline{php}{tests/Feature}.

Pri jednotlivých testoch sa nevyužíva MySQL databáza ako v prípade samotnej webovej aplikácie, ale databáza SQLite, ktorá je nakonfigurovaná tak, aby fungovala v pamäti počítača. Jej podrobnú konfiguráciu spolu s konfiguráciou testov je možné nájsť v súbore \mintinline{php}{phpunit.xml} nachádzajúci sa v hlavnom adresári webovej aplikácie.

\section{Overenie správnosti importu}
Nakoľko dáta museli byť pred samotným importom očistené z dôvodu ich nekonzistencie a duplicity, nebolo možné overiť správnosť importu oproti originálnym dátam. Preto som pristúpil ku kontrole importovaných dát voči očisteným dátam. Táto kontrola bola vykonaná porovnaním počtu záznamov importovaných a očistených dát

 Zo začiatku boli niektoré informácie zle importované --- táto skutočnosť sa týkala najmä previazanosti záznamov medzi jednotlivými tabuľkami. Následné opravy dát v databáze a v logike importu tieto problémy odstránili. Keďže dáta prešli takýmto náročnejším procesom importu, je možné, že niektoré dáta v databáze sa stále nebudú zhodovať s originálnymi dátami -- v tomto prípade musia byť dáta dodatočne manuálne upravené.

