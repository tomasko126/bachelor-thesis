\chapter{Analýza}
Táto kapitola sa venuje analýze súčasnej webovej aplikácie, spolu s jej architektúrou. Následne sú vymedzené jednotlivé funkčné a nefunkčné požiadavky novej webovej aplikácie, na ktoré nadväzuje sekcia používateľských rolí. Záver kapitoly je určený pre analýzu prípadov použití, ktoré vyplývajú z funkčných požiadaviek.

\section{Analýza súčasnej aplikácie}
== TO BE ADDED ==

\section{Analýza požiadaviek}
V tejto sekcii sa nachádza popis všetkých požiadaviek kladených na vznikajúcu webovú aplikáciu.
Tieto požiadavky delíme na funkčné a nefunkčné.

\subsection{Funkčné požiadavky}\label{funkcne-poziadavky}
Funkčné požiadavky vo všeobecnosti vymedzujú hranice aplikácie v kontexte jej funkcionality, ktorú používateľ aplikácie od systému očakáva. Ich úlohou je taktiež spresniť odhad pracnosti na vznikajúcej aplikácii.\cite{funkcne-poziadavky}

\subsubsection{Evidencia zvierat}\label{evidencia-zvierat}
Aplikácia musí umožňovať jednoduchú správu zvierat, od vytvorenia nového zvieraťa, cez jeho zobrazenia a editáciu, až po jeho prípadné zmazanie. V evidencii zvierat je pre každé zviera potrebné ukladať nasledujúce údaje:

\begin{itemize}
	\item meno a prezývku
	\item dátum narodenia
	\item majiteľa a chovateľa zvieraťa
	\item vrh, v ktorom sa zviera narodilo
	\item matku a otca zvieraťa
	\item pohlavie
	\item farba očí
	\item typ uší
	\item farba a typ srsti
	\item znaky
	\item dátum a dôvod úmrtia
\end{itemize}

Pre uľahčenie vytvorenia a editovania zvieraťa budú textové polia v maximálnej možnej miere interaktívne.

To znamená, že polia majiteľa a chovatela zvieraťa vrátia na základe vstupu zoznam majitelov, resp. chovatelov, ktorí už sú evidovaní v aplikácii. Následne si z tohto zoznamu používateľ zvolí požadovaného človeka.
V prípade, že aplikácia požadovaného človeka nevráti, používateľovi bude ponúknutá možnosť ho dodatočne vytvoriť.

Na podobnom princípe bude fungovať aj textové pole pre vrh, s tým rozdielom, že podľa zvoleného vrhu aplikácia predvyplní matku a otca zvieraťa.
 
\subsubsection{Evidencia registrácií zvieraťa}\label{evidencia-registracii-zvierata}
Je potrebné, aby aplikácia ponúkala možnosť zaregistrovať evidované zviera pod záujmový klub.
Pri takejto registrácií sa budú zbierať nasledovné údaje:

\begin{itemize}
	\item klub\footnote{Na výber z možností: ČKP, SOCHP alebo Ostatné}, pod ktorým bude zviera zaregistrované
	\item typ registrácie\footnote{Vyžadované iba v prípade registrácie zvieraťa pod klubom ČKP}
	\item registračné číslo
	\item dátum registrácie
	\item informáciu, či je nasledovný chov povolený
	\item informáciu o obmedzení chovu
\end{itemize}

\subsubsection{Evidencia vrhov}
Medzi nevyhnutnú funkcionalitu aplikácie sa radí aj evidencia vrhov. Tak ako pri evidencii zvierat, tak aj v tomto prípade používatelské rozhranie umožní vytvoriť nový vrh, zobraziť ho, respektíve ho editovať alebo vymazať. 

Pre účel evidencie vrhov sa budú v aplikácii ukladať nasledovné údaje:

\begin{itemize}
	\item typ a označenie vrhu
	\item majiteľ vrhu
	\item meno a kontakt na chovatela
	\item dátum narodenia
	\item matka a otec vrhu
	\item línia
	\item genetické informácie
	\item počet narodených a odchovaných mláďat
	\item počet odchovaných samčekov a samičiek
	\item počet mláďat určených pre maznanie a následný chov
\end{itemize}

Rovnako ako pri evidencii zvierat, tak aj tu umožní aplikácia interaktívne zvoliť matku, otca a majiteľa vrhu zobrazením zoznamu uložených zvierat/ľudí.

\subsubsection{Správa žiadostí o schválenie vrhov}\label{sprava-ziadosti-o-schvalenie-vrhu}
Pre organizáciu je žiaduce, aby aplikácia obsahovala správu žiadostí o schválenie vrhu.

V rámci žiadosti o schválenie vrhu rozlišujeme dva typy osôb -- žiadateľa o schválenie vrhu\footnote{Zväčša majiteľ daného vrhu}, a registrátora vrhu.

Aplikácia žiadatelovi umožní poslať žiadosť o schválenie vrhu s možnosťou zanechania poznámky pre registrátora priamo z prostredia konkrétneho vrhu.
Po odoslaní a úspešnom spracovaní tejto žiadosti serverom sa odošle e-mail všetkým registrátorom s informáciou o vytvorení novej žiadosti o schválení vrhu. Na tento e-mail bude môcť zareagovať akýkoľvek registrátor, a na základe dostupných informácií o vrhu sa rozhodne, či danú žiadosť schváli alebo zamietne.
O zmene stavu žiadosti bude žiadateľ informovaný e-mailom.

\subsubsection{Zobrazenie rodokmeňu zvieraťa a vrhu}\label{rodokmene}
Pre jednoduchšiu vizualizáciu predkov konkrétneho zvieraťa aplikácia ponúkne zobrazenie rodokmeňu zvieraťa vo forme jednoduchej tabuľky. V tejto tabuľke budú okrem mien zvierat zobrazené aj dodatočné informácie, definované v sekcii \ref{evidencia-zvierat}. Rodokmeň bude zobrazený v rámci jednotlivých zvierat a vrhov.

\subsubsection{Tvorba poznámok pre zviera a vrh}
V rámci aplikácie bude potrebné implementovať poznámky, ktoré budú môcť byť priradené jednotlivým zvieratám a vrhom.
Takáto poznámka by mala mať nastaviteľnú viditelnosť\footnote{Poznámka môže byť buď verejná alebo súkromná} a typ\footnote{Typ poznámky je jeden z nasledujúcich: všeobecná, upozornenie alebo výstraha}.
Taktiež poznámka musí poskytnúť informáciu, kedy bola vytvorená, respektíve editovaná.
Nakoľko sa jedná o poznámky pre zviera a vrh, budú zobrazené u príslušných zvieratách, resp. vrhoch, ku ktorým sa vzťahujú.

\subsubsection{Zobrazenie histórie zmien zvierat a vrhov}
Medzi želanú funkcionalitu novej webovej aplikácie patrí sledovanie a následné zobrazenie histórie zmien u všetkých zvierat a vrhoch.

Aplikácia bude zaznamenávať nasledujúce zmeny v systéme:
\begin{itemize}
	\item vytvorenie zvieraťa/vrhu
	\item úprava údajov u zvieraťa/vrhu
	\item zmazanie zvieraťa/vrhu
	\item obnova\footnote{Obnoviť zviera/vrh bude možné iba administrátor aplikácie} zmazaného zvieraťa/vrhu
\end{itemize}

Pri každej zmene popísanej vyššie je taktiež nutné ukladať, kto a kedy danú zmenu vykonal. V prípade úpravy údajov budú navyše zaznamenané tie údaje, ktoré boli zmenené používatelom. Túto históriu zmien bude možné vidieť vo forme tabuľky u každého zvieraťa, resp. vrhu.


\subsubsection{Generovanie preukazov}\label{generovanie-preukazov}
Pre potreby organizácie je nevyhnutné implementovať generovanie preukazov (osvedčení) vo forme PDF súboru. V tomto súbore bude prvá strana vyplnená informáciami o danom zvierati zo sekcie \ref{evidencia-zvierat}, vrátane jeho registrácie zo sekcie \ref{evidencia-registracii-zvierata}. V niektorých prípadoch bude taktiež zobrazená registrácia vrhu, ktorej obsah je definovaný v \ref{sprava-ziadosti-o-schvalenie-vrhu}, v ktorom sa zviera narodilo. Druhá strana súboru bude vyplnená rodokmeňom zvieraťa (\ref{rodokmene}).

Preukaz bude možné vygenerovať tlačidlom na stránke konkrétneho zvieraťa.

\subsubsection{Filtrovanie a radenie}
Pre vylepšenie používateľskej skúseností bude potrebné implementovať filtrovanie a radenie zvierat a ako aj vrhov na príslušných stránkach. To umožní jednoduchšiu prácu s aplikáciou a rýchlejšie nájdenie potrebných informácií.

\subsubsection{Správa používatelov a rolí}\label{sprava-pouzivatelov-a-roli}
Pre administrátorov aplikácie je nutné spravovať jednotlivých používateľov aplikácie, priradiť im príslušné role, alebo im ich naopak odoberať. Na základe tejto skutočnosti je žiaduce vytvoriť pohľad so zoznamom používateľov a ich rolami s následnou možnosťou im danú rolu zmeniť, prípadne daných používateľov odobrať.

\subsubsection{Lokalizácia aplikácie}
Nakoľko budú aplikáciu používať nie len českí ale aj zahraniční používatelia, je nutné, aby aplikácia poskytovala lokalizovaný obsah aj v anglickom jazyku. Jazyk bude možné jednoducho zmeniť v menu v paneli webovej aplikácie.

\subsection{Nefunkčné požiadavky}
Na rozdiel od funkčných požiadaviek, nefunkčné požiadavky umožňujú určiť obmedzenia kladené na aplikáciu. V druhom rade majú zásadný dopad na návrh architektúry webovej aplikácie.\cite{nefunkcne-poziadavky}

\subsubsection{Webová aplikácia}
Nakoľko je požadovaný systém navrhnutý ako webová aplikácia, bude potrebné, aby bola prístupná z internetu pomocou moderných webových prehliadačov.

\subsubsection{Používateľské rozhranie}
Webová aplikácia bude taktiež musieť obsahovať používateľské rozhranie, s ktorým budú môcť používatelia interagovať.
Prostredie bude naviac responzívne, čo uľahčí prípadný prístup do systému z mobilného prehliadača.

\subsubsection{Technológie}
Po konzultácii s vedúcim práce boli vymedzené nasledovné technológie, ktoré budú použité na strane servera.

Ako programovací jazyk bude použítý jazyk PHP, ktorý bol navrhnutý v roku 1994 Rasmusom Lerdorfom. Tento jazyk je univerzálny programovací jazyk na strane servera a je primárne určený na vývoj webových stránok.\cite{co-je-php}

Použitý jazyk bude vo verzii 7.3, hoci momentálne je najnovšia verzia jazyku 7.4\cite{verzie-php}.
Dôvod výberu nižšej verzie jazyka je daný PHP podporou webhostingu\footnote{Endora -- www.endora.cz}, na ktorom bude aplikácia nasadená.

Pre pridávanie, spracovanie a získavanie dát nám bude slúžiť open-source SQL databázový systém nazývaný MySQL, ktorý je vyvíjaný, distribuovaný a podporovaný spoločnosťou Oracle Corporation\cite{co-je-mysql}. Tento databázový systém budeme používať vo verzii 5.6.

Táto verzia MySQL taktiež nie je najnovšia (v skutočnosti bola prvýkrát vydaná v roku 2013, avšak je stále oficiálne podporovaná\cite{verzie-mysql}), opäť z dôvodu neexistujúcej podpory novšieho databázového systému webhostingom.

\section{Používateľské role}\label{pouzivatelske-role}
Novovznikajúca webová aplikácia bude prístupná iba zaregistrovaným a prihláseným používateľom. Navyše, niektoré akcie budú obmedzené iba pre určitý okruh používateľov.

K tomu, aby sme umožnili vybraným používateľom umožniť alebo zakázať vybrané akcie, bude potrebné do aplikácie implementovať používateľské role a práva. Následne bude aplikácia riadiť prístup používateľa k jednotlivým akciám na základe príslušnosti k vybranej roli.

V nasledujúcich podsekciách budú priblížené jednotlivé role a k nim príslušné práva, ktoré sa budú vyskytovať v aplikácii.

\subsection{Bežný používateľ}
Túto rolu má každý používateľ automaticky po registrácii do webovej aplikácie. Bežný používateľ môže v aplikácii: 

\begin{itemize}
	\item Vytvoriť a zobraziť svoje zvieratá
	\item Upraviť svoje zvieratá v prípade, že nie sú zaregistrované pod klubom ČKP
	\item Vytvoriť a upraviť registrácie popísané v \ref{evidencia-registracii-zvierata} svojich vlastných zvierat, ktoré nespadajú pod klub ČKP
	\item Zobraziť zoznam všetkých cudzích zvierat spolu s ich detailmi
	\item Vytvoriť a zobraziť vrhy, u ktorých je používateľ ich majiteľom
	\item Upraviť vrhy, ktorých je majitelom, pokiaľ neboli tieto vrhy schválené žiadosťou popísanou v \ref{sprava-ziadosti-o-schvalenie-vrhu}
	\item Zobraziť všetky vrhy typu VP alebo schválené vrhy typu PP a NV spolu s ich detailmi
	\item Pridať poznámku k zvieratám a vrhom, ktoré vlastní
\end{itemize}

\subsection{Registrátor zvierat}
Registrátor zvierat je v hierarchii rolí postavený nad bežným používateľom. Tým pádom má všetky práva bežného používateľa a navyše nasledujúce práva:

\begin{itemize}
	\item Možnosť pridať poznámku k akýmkoľvek zvieratám
	\item Možnosť pridať, editovať a vymazať akúkoľvek registráciu u každého zvieraťa
\end{itemize} 

\subsection{Schvalovateľ vrhov}
Schvalovateľ vrhov je taktiež postavený v hierarchii rolí nad bežným používateľom podobne ako registrátor zvierat, s tým rozdielom, že schvalovateľ vrhu môže v aplikácii:

\begin{itemize}
	\item Vidieť a odpovedať na žiadosti o schválenie vrhov definované v \ref{sprava-ziadosti-o-schvalenie-vrhu}
	\item Editovať akékoľvek zviera a vrh
	\item Pridať poznámky k akémukoľvek vrhu
	\item Vygenerovať preukaz zvieraťa popísaný v \ref{generovanie-preukazov}
\end{itemize}

\subsection{Administrátor}
Ako obvykle, pre administrátora neplatia žiadne reštrikcie, čo znamená, že bude mať prístup ku všetkej funkcionalite definovanej vo funkčných požiadavkách v sekcii \ref{funkcne-poziadavky}.

\section{Prípady použitia}
Táto sekcia sa venuje opisom prípadov použití -- čo sú špecifikácie rôznych činností, ktoré môžu používatelia s aplikáciou vykonávať.\cite{co-su-pripady-pouzitia} Tieto prípady použitia budú zachytené vo forme scenáru a budú vychádzať nielen zo známych funkčných požiadavok popísaných v sekcii \ref{funkcne-poziadavky}, ale aj z jednotlivých používateľských rolí, ktoré boli definované v sekcii \ref{pouzivatelske-role}. V tejto práci som sa rozhodol venovať iba takým prípadom použitia, ktoré sú z môjho pohľadu pre čitateľa prínosnejšie. Tie som následne rozdelil podľa príslušnosti k jednotlivým prvkom aplikácie. Taktiež všetky nasledujúce prípady použitia predpokladajú, že používateľ bol do systému úspešne prihlásený.

\subsection{Prípady použitia týkajúce sa zvierat}

\subsubsection*{UC1 -- Vyhľadanie zvieraťa}\label{uc1}

Vyhľadanie zvieraťa je jeden zo základných prípadov použitia, kedy používateľ chce vyhladať dané zviera.

\subsubsection*{Scenár}

\begin{enumerate}
	\item Používateľ zvolí možnosť \uv{Zvieratá} z navigačného menu aplikácie.
	\item Následne aplikácia zobrazí tabuľku so zvieratami a filtrom, na základe ktorého si používateľ môže nastaviť dodatočné kritéria filtrovania zvierat.
	\item Používateľ nastaví filter podľa kritérií hľadaného zvieraťa a klikne na tlačidlo \uv{Filtrovať}.
	\item Aplikácia zobrazí všetky zvieratá vyhovujúce zadaným kritériám.
\end{enumerate}

\subsubsection*{Alternatívny scenár}

\begin{enumerate}
    \item [4.] Aplikácia zobrazí hľadané zviera už po načítaní tabuľky so zvieratami. V tomto prípade scenár vyhľadávania zvieraťa končí.
\end{enumerate}

Tento prípad použitia realizuje funkčné požiadavky na aplikáciu, konkrétne \hyperref[evidencia-zvierat]{Evidenciu zvierat}.

\subsubsection*{UC2 -- Zobrazenie detailu zvieraťa}\label{uc2}

Tento prípad použitia popisuje zobrazenie detailu zvieraťa a zahŕňa prípad \nameref{uc1}.

\subsubsection*{Scenár}

\begin{enumerate}
	\item UC1 Vyhľadanie zvieraťa
	\item Používateľ klikne na meno zvieraťa, ktorého detail si želá vidieť
	\item Aplikácia zobrazí detail zvieraťa
\end{enumerate}

Popísaný prípad taktiež realizuje funkčnú požiadavku na aplikáciu, a to \hyperref[evidencia-zvierat]{Evidenciu zvierat}.

\subsection*{UC3 -- Vytvorenie zvieraťa}

Medzi ďalší základný prípad použitia patrí pridanie zvieraťa používatelom do systému.

\subsubsection*{Scenár}

\begin{enumerate}
	\item Používateľ navštívi domovskú stránku aplikácie.
	\item Následne zvolí možnosť \uv{Zvieratá} z navigačného menu aplikácie.
	\item Po načítaní stránky používateľ klikne na tlačidlo \uv{Pridať zviera}.
	\item Aplikácia načíta novú stránku s formulárom pre vytvorenie nového zvieraťa.
	\item Používateľ vyplní formulár.
	\item Po vyplnení všetkých potrebných údajov pre vytvorenie zvieraťa aplikácia umožní odoslať formulár kliknutím na tlačidlo \uv{Uložiť}
	\item Používateľ odošle vyplnený formulár kliknutím na tlačidlo \uv{Uložiť}.
	\item Po úspešnom vytvorení zvieraťa na základe vložených údajov systém presmeruje používateľa na stránku so zoznamom uložených zvierat.
\end{enumerate}

Tento prípad použitia realizuje tú istú funkčnú požiadavku na aplikáciu a predošlé prípady použitia -- konkrétne \hyperref[evidencia-zvierat]{Evidenciu zvierat}.

\subsection*{UC4 -- Editácia zvieraťa}

V tomto prípade sa používateľ aplikácie snaží upraviť existujúce zviera v systéme. Tento prípad použitia taktiež zahŕňa prípady 
\nameref{uc1} alebo \nameref{uc2}, v závislosti od spôsobu editácie zvieraťa, ktorý si používateľ zvolí.

\subsubsection*{Scenár}

\begin{enumerate}
	\item UC1 Vyhľadanie zvieraťa
	\item Používateľ klikne na rozbalovacie menu u zvieraťa, ktoré chce editovať.
	\item Po kliknutí na rozbalovacie menu, aplikácia ponúkne možnosť editácie alebo zmazanie zvieraťa.
	\item Používateľ zvolí možnosť editácie.
	\item Aplikácia následne presmeruje používatela na stránku s editáciou zvieraťa s predvyplnenými údajmi.
	\item Používateľ upraví údaje zvieraťa.
	\item Kliknutím na tlačidlo \uv{Odoslať} sa odošle vyplnený formulár na server.
	\item Po úspešnom spracovaní formuláru je používateľ presmerovaný na stránku s detailmi upravovaného zvieraťa.
\end{enumerate}

\subsubsection*{Alternatívny scenár}

\begin{enumerate}
	\item UC2 Zobrazenie detailu zvieraťa
	\item Používateľ klikne na tlačidlo \uv{Upraviť} na pravom postrannom paneli.
	\item Následne aplikácia presmeruje používateľa na stránku s editáciou zvieraťa s predvyplneným formulárom obsahujúcim údaje o editovanom zvierati.
	\item Používateľ upraví požadované údaje zvieraťa.
	\item Kliknutím na tlačidlo \uv{Odoslať} sa odošle vyplnený formulár na server.
	\item Po úspešnom spracovaní formuláru je používateľ presmerovaný na stránku s detailmi upravovaného zvieraťa.
\end{enumerate}

Popísaný prípad použitia realizuje funkčnú požiadavku \hyperref[evidencia-zvierat]{Evidencie zvieraťa}.

\subsection*{UC5 -- Vytvorenie registrácie zvieraťa}

\subsection{Prípady použitia týkajúce sa vrhov}

\subsection*{UC6 -- Vytvorenie žiadosti o schválenie vrhu}

\subsection*{UC7 -- Odpoveď na žiadosť o schválenie vrhu}











