\chapter{Implementácia}
Cieľom tejto kapitoly je na základe predošlej analýzy súčasnej aplikácie a návrhu databázy priblížiť implementačnú časť novej aplikácie. Najprv sa kapitola venuje implementácii aplikácie na serverovej časti s priblížením architektonického vzoru MVC, ktorý sa využíva pri tvorbe webových aplikácií. Následujúca sekcia je vymedzená implementácii klientskej časti aplikácie. Záver je venovaný migrácii dát z pôvodnej do novej aplikácie.

\section{Serverová časť}
Implementácia serverovej časti webovej aplikácie sprevádzala využitie zvoleného PHP frameworku, ktorým bol Laravel.

\subsection{Štruktúra projektu}
Pre inicializáciu nového Laravel projektu som použil príkaz\\ \mintinline{text}{laravel new pedigree-book}. Tento príkaz vytvoril nasledujúcu adresárovú štruktúru.

\begin{figure}[H]
	\dirtree{%
		.1 app \DTcomment{Adresár určený pre implementáciu aplikácie}.
		.1 bootstrap \DTcomment{Adresár určený pre aplikačnú cache}.
		.1 config \DTcomment{Adresár obsahujúci konfiguračné súbory aplikácie}.
		.1 database \DTcomment{Adresár určený pre databázové migrácie, továrne a seeds}.
		.1 public \DTcomment{Adresár určený pre všetky verejne dostupné súbory}.
		.1 resources \DTcomment{Adresár určený pre klientské skripty a Vue.js elementy}.
		.1 routes \DTcomment{Adresár obsahujúci router}.
		.1 storage \DTcomment{Adresár určený pre súbory generované frameworkom}.
		.1 tests \DTcomment{Adresár určený pre aplikačné testy}.
		.1 vendor \DTcomment{Adresár obsahujúci nainštalované serverové balíčky}.
	}
\end{figure}

Po spustení príkazu \mintinline{php}{php artisan serve} sa spustil server \\ na adrese \mintinline{php}{localhost:8000}. Po navštívení tejto adresy vo webovom prehliadači bolo možné vidieť predvolenú úvodnú stránku Laravelu.

\subsection{MVC architektúra}
Laravel je implementovaný na základe architektonického vzoru MVC, ktorý je akronymom pre Model-View-Controller. Použitie tohto vzoru rozdeľuje aplikáciu do troch hlavných skupín --- na modely, views a controllery. Nasledujúce podsekcie vysvetlia význam jednotlivých skupín.

\subsubsection*{Model}
Model (modelová trieda) obsahuje logiku aplikácie a všetko, čo do tejto logiky spadá. Môžu to byť rôzne výpočty, databázové dotazy, validácia prichádzajúcich dát a podobne. Funkcia modelu spočíva v prijatí parametrov, ktoré spracuje s možným využitím dát z databázy. Následne takto spracované dáta vráti späť metóde, ktorá danú funkciu modelu volala \cite{mvc-architektura}.

Platí, že pre každú tabuľku v databáze by mala byť vytvorená príslušná modelová trieda. Takáto modelová trieda je následne úzko spätá s danou tabuľkou, nakoľko poskytuje rôzne metódy pre prácu s dátami uloženými v tejto tabuľke \cite{co-je-eloquent}.

Laravel pristupuje k týmto dátam pomocou frameworku Eloquent, ktorý je určený pre objektovo-relačné mapovanie.
Objektovo-relačné mapovanie (ORM) je technika prístupu k dátam uloženým v databáze. Táto technika spočíva v mapovaní riadkov príslušnej tabuľky v relačnej databáze do objektov, ktoré sú instancie jednotlivých modelov. Táto technika funguje aj opačným smerom.

\begin{figure}[H]
\begin{minipage}[]{\linewidth}
\begin{minted}[xleftmargin=1.8em,linenos]{php}
/**
 * Create new animal
 * @param $data
 * @return Animal
 * @throws Throwable
 */
public static function createAnimal($data) {
    $animal = new Animal($data);
    $animal->creator_id = Auth::id();
    $animal->saveOrFail();
    return $animal->refresh();
}
\end{minted}
\end{minipage}
\caption[Ukážka modelovej triedy Animal.php]
{Ukážka modelovej triedy Animal.php}
\label{animal-code-example}
\end{figure}
Na obrázku \ref{animal-code-example} je možné vidieť metódu \mintinline{php}{createAnimal}, ktorá je zodpovedná za vytvorenie nového zvieraťa. Táto metóda sa nachádza v modelovej triede \mintinline{php}{Animal}, ktorá je modelovou triedou tabuľky \mintinline{php}{animals}. V prvom riadku tela metódy prebieha vytvorenie novej instancie modelu, ktorá je naplnená dátami nachádzajúcimi sa v parametri \mintinline{php}{$data}. Následne je parameter \mintinline{php}{creator_id} objektu \mintinline{php}{Animal} nastavený na identifikátora používateľa, od ktorého požiadavka na vytvorenie nového zvieraťa prišla. Nie je náhoda, že názov tohto parametru je možné nájsť ako stĺpec v tabuľke \mintinline{php}{animals}. Zavolaním Eloquent metódy \mintinline{php}{saveOrFail} sa daný objekt pretransformuje do SQL príkazu určeného pre pridanie dát do tabuľky. Následne sa tento príkaz vykoná na pozadí. V prípade, že uloženie objektu do databázy z rozličných dôvodov zlyhá, táto metóda vráti výnimku. Nakoniec sa zavolaním metódy \mintinline{php}{refresh} aktualizuje daná instancia modelu aktuálnymi dátami z databázy (o identifikátor zvieraťa a systémové stĺpce).

\pagebreak

\subsubsection*{View}
View je časť aplikácie, ktorá zodpovedá za vykreslenie obsahu používateľom. Jej obsahom je šablóna, obsahujúca HTML stránku a tagy nejakého značkovacieho jazyka, ktorý umožňuje do šablóny vkladať premenné, prípadne vykonávať iterácie a podmienky. View nie je len šablóna, ale zobrazovač výstupu obsahujúca minimálne množstvo logiky potrebnej pre výpis dát.
Podobne ako model, view nevie, odkiaľ mu dáta prišli, iba sa stará o ich finálne zobrazenie používateľovi \cite{mvc-architektura}.

V prípade Laravelu sa používa v šablónach značkovací jazyk Blade. Blade umožňuje nie len vytvárať jednotlivé šablóny, ale aj komponenty, ktoré môžu byť použité viackrát v rôznych šablónach. Tieto šablóny sa nachádzajú v adresári \mintinline{php}{resources/views.}

\begin{figure}[H]
\begin{minipage}[]{\linewidth}
\begin{minted}[xleftmargin=1.8em,linenos]{php}
<section id="breeding-info" class="bold">
    <span class="breeding_available">
        @isset($isAnimalAvailableForBreeding)
            {{ $isAnimalAvailableForBreeding ?
               'Chov povolen' : 'Chov není povolen' 
            }}
        @endisset
    </span>
    <p class="breeding_limitation">
        @isset($animalBreedingLimitation)
        {{ !empty($animalBreedingLimitation()) ?
           ('Poznámka: ' . $animalBreedingLimitation) : '' 
        }}
        @endisset
    </p>
</section>
\end{minted}
\end{minipage}
\caption[Ukážka šablóny animal\_overview.blade.php]
{Ukážka šablóny animal\_overview.blade.php}
\label{animal-overview-view-code}
\end{figure}

Na obrázku \ref{animal-overview-view-code} sa nachádza ukážka časti šablóny, ktorá je zodpovedná za vygenerovanie preukazu obsahujúci detaily zvieraťa. Ako si je možné všimnúť, obsah tejto šablóny sa skladá z použitia HTML jazyka pre definovanie štruktúry preukazu, zo špeciálnych Blade príkazov ako \mintinline{text}{isset} a \mintinline{php}{endisset}, a z funkcie s premennými v syntaxe jazyka PHP.\pagebreak

\subsubsection*{Controller}
Controller je poslednou časťou MVC, ktorá ozrejmí funkčnosť tohto vzoru. Jedná se o prostredníka, s ktorým komunikuje používateľ, model i view. Tieto komponenty drží pohromade a zabezpečuje komunikáciu medzi nimi \cite{mvc-architektura}. Samotné controllery aplikácie sa nachádzajú v adresári \mintinline{php}{app/Http/Controllers}.

\begin{figure}[H]
\begin{minted}[xleftmargin=1.8em,linenos]{php}
/**
 * Show view with animal's template for .pdf export
 *
 * @param int $id
 * @return Factory|View
 * @throws AuthorizationException
 */
public function export(int $id) {
    $animal = Animal::getAnimalForPdf($id);

    // Check, whether user is authorized to view this page
    $this->authorize('canExportAnimal', $animal);

    $litter = $genealogy = null;

    // Is animal part of some litter?
    if (isset($animal->litter_id)) {
        $litter = Litter::getLitter($animal->litter_id);
        $genealogy = Animal::getGenealogy($litter, 0);
    }
    
    (...)
        
    return view('generated_docs.pdf', [
        'animal' => $animal,
        'litter' => $litter,
        'genealogy' => $genealogy,
        'orientation' => $orientation
    ]);
}
\end{minted}
\caption[Ukážka controlleru AnimalController.php]
{Ukážka controlleru AnimalController.php}
\label{animal-controller-view-code}
\end{figure}

Na obrázku \ref{animal-controller-view-code} sa nachádza ukážka metódy \mintinline{php}{export}. Z obrázku je možné vidieť volanie metód modelov \mintinline{php}{Animal} a \mintinline{php}{Litter}. Po tom čo controller dostane dáta z jednotlivých metód modelov, sú tieto dáta predané view, ktorá ich následne doplní do šablóny \mintinline{php}{pdf}. Nakoniec controller šablónu s dátami vráti, a framework sa postará o jej odoslanie do klientského prehliadača, ktorý danú šablónu zobrazí používateľovi.

Avšak, na akom základe sa spustí táto metóda? Odpoveďou na túto otázku je časť aplikácie zvaná \uv{Router}.
Router je zodpovedný za spustenie metód controllerov na základe prijatého HTTP požiadavku (buď iniciovaného používateľom navštívením danej adresy alebo skriptom, ktorý pošle požiadavku na danú adresu). Na základe URL adresy požiadavku router rozhodne, ktorú metódu daného controlleru spustí.

Router aplikácie sa nachádza v priečinku \mintinline{php}{routes}.

\begin{figure}[H]
\begin{minted}[xleftmargin=1.8em,linenos]{php}
Route::middleware('auth:sanctum')->group(function() {
    Route::get(
    '/animals/{id}/export',
    'AnimalsController@export'
    )->name('animals.id.export.get');
    
    (...)
});
\end{minted}
\caption[Ukážka routeru web.php]
{Ukážka routeru web.php}
\label{router-code}
\end{figure}

Na obrázku \ref{router-code} je zobrazená ukážka routeru. V prípade, ak HTTP požiadavka prichádzajúca na server bude mať adresu v tvare\\ \uv{https://domena.sk/animals/<id>/export}\footnote{<id> sa nahradí ľubovoľným číslom}, router spustí metódu \mintinline{php}{export} v controlleri \mintinline{php}{AnimalsController}.

\section{Klientská časť}
Klientská časť aplikácie sa skladá prevažne z implementácie jednotlivých stránok, ktoré sa skladajú z viacerých nezávislých komponent, ktoré spolu komunikujú. Tieto stránky sú zobrazované klientskym frameworkom Vue.js, ktorý sa stará o správne zobrazenie jednotlivých stránok na základe adresy, na ktorej sa používateľ nachádza.

Spustením príkazu \mintinline{php}{npm install vue} sa nainštaloval framework Vue.js spolu s jej závislosťami do adresára \mintinline{php}{node_modules}, ktoré sú potrebné pre beh samotného frameworku.

Komponent, ktorý zobrazí stránku zvieraťa, vyzerá nasledovne:

\begin{figure}[H]
\begin{minted}[xleftmargin=1.8em,linenos]{html}
<template>
    <div id="animal-page" class="columns section">
        <left-panel
            :animal="animal"
            :user="user">
        </left-panel>
        <animal-information
            :animal-id="animalId"
            :animal="animal"
            :is-loading="isLoading"
            :user="user">
        </animal-information>
        <right-panel
            :animal-id="animalId"
            :is-loading="isLoading"
            :animal="animal"
            :user="user"
            :key="animalId">
        </right-panel>
    </div>
</template>
\end{minted}
\caption[Ukážka šablóny komponentu AnimalPage.vue]
{Ukážka šablóny komponentu AnimalPage.vue}
\label{animal-page-vue-code}
\end{figure}

Komponenty vo Vue.js sa definujú pomocou štandardných HTML značiek. Okrem iného jednotlivým komponentom môžu byť predané premenné rôzneho typu. Vue.js taktiež podporuje vkladanie existujúcich komponentov do výsledného komponentu. Týmto spôosobom sa dá dosiahnuť znovupoužiteľnosť jednotlivých komponentov naprieč veľkým množstvom stránok.

V horeuvedenom prípade komponent \mintinline{php}{AnimalPage} obsahuje tri komponenty:
\begin{itemize}
	\item LeftPanel
	\item AnimalInformation
	\item RightPanel
\end{itemize}

Samotný komponent obsahuje nie len jej deklaráciu v rámci HTML jazyka, ale aj biznis logiku v jazyku JavaScript. Väčšinou sa jedná o metódy, ktoré buď načítavajú dáta zo servera, alebo reagujú na používateľský vstup.

V prípade \mintinline{php}{AnimalPage} komponentu sa jedná o nasledujúcu logiku:
\raggedbottom

\begin{figure}[H]
\begin{minted}[xleftmargin=1.8em,linenos]{js}
<script>
    export default {
        name: "AnimalPage",
        components: {RightPanel, LeftPanel, AnimalInformation},
        data() {
            return {
                animal: null,
                isLoading: false,
            }
        },
        computed: {
            animalId() {
                return Number(this.$route.params.animal);
            }
        },
        methods: {
            async loadData() {
                this.isLoading = true;
                const url = `/api/animals/${this.animalId}`;
                try {
                    const request = await axios.get(url);
                    this.animal = request.data;
                } catch (e) {
                    this.$buefy.toast.open({
                        message:
                        this.$t('animal.index.page_load_fail'),
                        type: 'is-danger' 
                     });
                    throw e;
                } finally {
                    this.isLoading = false;
                }
            }
        },
        async created() {
            await this.loadData();
        }
    }
</script>
\end{minted}
\caption[Ukážka biznis logiky AnimalPage.vue komponentu]
{Ukážka biznis logiky AnimalPage.vue komponentu}
\label{animal-page-business-code}
\end{figure}
\pagebreak
Hodnota atribútu \mintinline{php}{name} určuje meno daného komponentu naprieč celou klientskou aplikáciou.
Obsah objektu \mintinline{php}{components} deklaruje použité komponenty v rámci \mintinline{php}{AnimalPage}.
Metóda \mintinline{php}{data} vracia objekt s reaktívnymi atribútami --- pri zmene jednotlivých atribútov Vue.js notifikuje komponent o zmene atribútu. V prípade, že sa daný atribút používa v komponente, Vue.js zaistí jeho znovuvykreslenie.
Dané atribúty sa taktiež používajú na ukladanie stavu jednotlivých komponentov.
Objekt \mintinline{php}{methods} obsahuje biznis metódy, ktoré môžu byť volané inými metódami. V tomto prípade sa používateľská metóda \mintinline{php}{loadData} spustí po tom, čo sa spustí exekúcia metódy \mintinline{php}{created}. Metóda \mintinline{php}{created} je ale systémová -- konkrétne sa exekuuje po vytvorení daného komponentu na stránke.

Zjednodušený životný cyklus tohto komponentu bude nasledovný\footnote{Informácie o kompletnom životnom cykle komponentov je možné nájsť na https://vuejs.org/v2/guide/instance.html}:
\begin{enumerate}
	\item Vue.js vytvorí a zobrazí komponent \mintinline{php}{AnimalPage}.
	\item Po vytvorení tohto komponentu bude automaticky spustená metóda \mintinline{php}{created}, ktorá vzápätí zavolá metódu \mintinline{php}{loadData}.
	\item Daná metóda sa postará o načítanie informácii o zvierati s identifikátorom \mintinline{php}{animalId}.
	\item V prípade zlyhania načítavania dát sa zobrazí chybová hláška, v opačnom prípade sa informácie o zvierati uložia do premennej \mintinline{php}{animal}.
	\item Obsah premennej \mintinline{php}{animal} sa spropaguje do komponentov, ktoré prijímajú objekt \mintinline{php}{animal}.
	\item Ak daný komponent závisí pri jeho zobrazení od obsahu premennej \mintinline{php}{animal}, bude znovu vykreslený.
\end{enumerate}

Po zmene jednotlivých komponentov je potrebné spustiť príkaz \mintinline{php}{npm run prod}, ktorý jednotlivé komponenty skompiluje do formy zrozumiteľnej prehliadaču. Pri vývoji aplikácie je však komfortnejšie použiť príkaz \mintinline{php}{npm run watch}, ktorý na pozadí zaistí automatickú kompiláciu komponentov po každej ich zmene. Po následnom znovunačítaní stránky prehliadača (\mintinline{php}{localhost:8000}) sa zobrazia najnovšie zmeny aplikácie.

\pagebreak

\section{Migrácia dát}
Po implementácii novej aplikácie bolo potrebné zmigrovať existujúce dáta do novej aplikácie. Proces migrácie spočíval vo vytvorení starých tabuliek s dátami v databáze novej aplikácie. Následne boli pre tieto tabuľky vytvorené modelové triedy, ktoré obsahujú logiku pre transformáciu dát zo starých do nových tabuliek. Tieto modelové triedy je možné nájsť v priečinku \mintinline{php}{app/ImportModels}.

Každá modelová trieda obsahuje metódu \mintinline{php}{import}. Tieto metódy sú volané v jednotlivých metódach controllera \mintinline{php}{ImportController}, ktoré sú zaregistrované v routri \mintinline{php}{web.php} \ref{router-import-code}.

\begin{figure}[H]
\begin{minted}[xleftmargin=1.8em,linenos]{html}
Route::get(
    '/import/contacts',
    'ImportController@importContacts'
)->name('import.contacts.get');
Route::get(
    '/import/czkp_animals',
    'ImportController@importCZKPAnimals'
)->name('import.czkp_animals.get');
Route::get(
    '/import/czkp_litters',
    'ImportController@importCZKPLitters'
)->name('import.czkp_litters.get');
Route::get(
    '/import/czkp_animals/litters',
    'ImportController@connectAnimalWithLitter'
)->name('import.czkp_animals.litters.get');
Route::get(
    '/import/czkp_babies',
    'ImportController@importCZKPBabies'
)->name('import.czkp_babies.get');
Route::get(
    '/import/pp_information',
    'ImportController@importPPInformation'
)->name('import.pp_information.get');
\end{minted}
\caption[Ukážka volania import metód v routri web.php]
{Ukážka volania import metód v routri web.php}
\label{router-import-code}
\end{figure}

Pred importom dát muselo byť veľké množstvo záznamov manuálne očistených, z dôvodu nekonzistentnosti mien majiteľov/chovateľov, ich kontaktných údajov a príslušnosti k jednotlivým chovným staniciam medzi súčasnými tabuľkami. Táto nekonzistentnosť dát bola spôsobená ich ukladaním do nenormalizovaných tabuliek. Čistenie dát zahŕňalo manuálnu kontrolu a dodatočnú úpravu vyše 3500 záznamov v súčasnej databáze.

Ak by sa záznamy neočistili, nebolo by možné nie len spárovať jednotlivé zvieratá a vrhy k ich majiteľom a chovateľom, ale ani importovať týchto ľudí do tabuľky \mintinline{php}{people} a spárovať ich s jednotlivými chovnými stanicami z tabuľky \mintinline{php}{stations}.
