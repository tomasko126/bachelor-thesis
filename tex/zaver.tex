\begin{conclusion}
V analytickej časti tejto práce som sa zameral na analýzu súčasnej aplikácie, prevažne z pohľadu jej architektúry a štruktúry používanej databázy. Následne som zozbieral funkčné a nefunkčné požiadavky na základe funkcionality doterajšej aplikácie. Nakoľko aplikáciu budú používať ľudia s rôznymi právami k rôznym častiam aplikácie, bolo potrebné implementovať správu rolí a práv. Nadväzujúce prípady použitia definovali presnejšiu podobu implementovanej aplikácie.

V časti návrhu aplikácie som porovnal MPA a SPA architektúru, ktoré sa používajú pri implementácii webových aplikácií. Na základe ich porovnania a doterajších skúseností bola vybraná SPA architektúra, ktorá spolu s následnou voľkou technológií bola implementačným základom pre vznikajúcu aplikáciu. Okrem výberu architektúry a voľby použítých technológií bolo nevyhnutné normalizovať tabuľky v existujúcej databáze. Tieto normalizované tabuľky poslúžili ako základ navrhnutých tabuliek v novom databázovom modeli aplikácie, ktorý bol navrhnutý s cieľom predísť duplikácii dát.

Implementačná časť obsahuje detaily ohľadom implementácie serverovej a klientskej časti, s bližším popisom MVC architektúry.
Okrem samotnej implementácie bolo nutné premigrovať existujúce dáta do novej databázy, aby aplikáciu mohli ihneď začať používať existujúci používatelia starej aplikácie. Nakoľko boli dáta z veľkého množstva v existujúcej databáze duplicitné, a ich očistenie nebolo možné zautomatizovať, musela byť veľká časť dát očistená manuálne.

Po implementácii samotnej aplikácie nasledovalo jej otestovanie, aby sa overilo, či aplikácia neobsahuje implementačné chyby a či zodpovedá zadaným požiadavkám. V neposlednom rade nasledovala kontrola správnosti importu existujúcich dát do databázy novej aplikácie. Keďže originálne dáta museli byť pred importom očistené, nebolo možné s istotou overiť správnosť importu týchto dát voči originálnym dátam.

Výsledná webová aplikácia pokrýva všetky funkčné a nefunkčné požiadavky, ktoré boli na ňu kladené v analytickej časti práce.
\end{conclusion}