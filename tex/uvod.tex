\begin{introduction}

Používanie webových aplikácií sa stalo bežnou súčasťou našich životov vďaka rozšírenému používaniu mobilných zariadení schopných pripojiť sa na internet. Ich používatelia nie sú len bežní ľudia, ale aj záujmové organizácie, medzi ktoré taktiež patrí organizácia \uv{Český klub potkanů}. Táto organizácia používa webovú aplikáciu pre evidenciu informácií o zvieratách v aktívnom chove a ich vrhoch, pomáha jej členom v plánovaní odchovov a umožňuje pre ne generovať jednoduché preukazy. Avšak problémom existujúcej aplikácie je jej zastaranosť, ťažkopádnosť pri jej používaní a použitie zlých návrhových vzorov pri jej vytváraní.

Táto bakalárska práca sa zaoberá vytvorením novej webovej aplikácie, ktorá nahradí starú aplikáciu a tým uľahčí organizácii evidenciu jednotlivých informácií.

Rešeršná časť práce sa venuje analýzou existujúcej aplikácie z pohľadu architektúry aplikácie a jej databázy. Na túto časť nadviaže analýza požiadaviek s popisom prípadov použití, ktoré poslúžia ako základ implementácie jednotlivých funkcií.
Následne sa práca zaoberá zvolením vhodnej architektúry potrebnej pre beh aplikácie, technológií, ktoré aplikácia bude používať a návrhom databázy, ktorá vyrieši existujúci problém ukladania duplicitných dát.

Implementačná časť práce sa venuje procesu implementácie aplikácie z pohľadu serverovej a klientskej časti. Následne pokračuje opisom procesu migrácie dát z existujúcej databázy do novej databázy navrhnutej pre novú aplikáciu.

\pagebreak

Práca sa taktiež venuje testovaniu aplikácie pomocou integračných testov, ktoré overia, či jej implementácia prebehla korektne podľa požiadaviek definovaných v analytickej časti práce. V neposlednom rade sa práca zaoberá kontrolou importovaných existujúcich dát do databázy implementovanej aplikácie.

\end{introduction}
